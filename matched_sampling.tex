\documentclass{article} % For LaTeX2e
\usepackage{nips15submit_e,times}
\usepackage{natbib}
%\documentstyle[nips13submit_09,times,art10]{article} % For LaTeX 2.09

%python 2.7.3

\usepackage{graphicx}
\usepackage{amsmath}
\usepackage{amssymb}
\usepackage{verbatim}
\usepackage[multiple]{footmisc}

\makeatletter
\@addtoreset{subsubsection}{section}
\makeatother


\title{Optimal and implicit baselines for belief networks}

%\author{Jason Tyler Rolfe, Kyunghyun Cho, Aaron Courville, \& Yoshua Bengio \\
%{\bf \today}} 

\author{
Jason Tyler Rolfe \\ %\thanks{ Use footnote for providing further information about author (webpage, alternative address)---\emph{not} for acknowledging funding agencies.} \\
D-Wave Systems \\
\texttt{jrolfe@dwavesys.com} \\
\And
Kyunghyun Cho \\
Universite de Montreal \\
\texttt{kyunghyun.cho@montreal.ca} \\
\AND
Aaron Courville \\
Universite de Montreal \\
\texttt{aaron.courville@umontreal.ca} \\
\And
Yoshua Bengio \\
Universite de Montreal \\
\texttt{yoshua.bengio@gmail.com} \\
}


%\date{\today}

\def\etal{{\textit{et~al.~}}}
\def\normal{\mathcal{N}}
\def\KL{\text{KL}}
\def\cov{\text{cov}}

\newcommand\pderiv[2]{\ensuremath \frac{\partial #1}{\partial #2}}
\newcommand\ex{\mathbb{E}}

\def\l||{\left|\left|}
\def\r||{\right|\right|}
%\def\l|{\left|}
%\def\r|{\right|}

\DeclareMathOperator{\Tr}{Tr}

\newcommand{\fix}{\marginpar{FIX}}
\newcommand{\new}{\marginpar{NEW}}

\nipsfinalcopy % Uncomment for camera-ready version


\begin{document}

\maketitle

\begin{abstract}
Variational expectation maximization with directed graphical models requires only tractable sampling operations to produce unbiased estimates of the gradient of the expectation lower bound.  While the variance of these unbiased estimates is small for continuous distributions that can make use of the variational auto-encoder formulation, the variance is much larger for discrete distributions.  We analyze the variance of the gradient using techniques from the deep learning and reinforcement learning literature, and introduce a new approach that achieves lower variance with fewer samples.  
\end{abstract}

\section{Introduction}


Unsupervised learning of probabilistic models has proven to be a powerful technique for machine learning, facilitating tasks such as denoising and inpainting, and regularizing supervised tasks such as classification \cite{hinton2006fast, salakhutdinov2009deep}.  Conventionally, the log-likelihood of an observed dataset is maximized, or equivalently the KL-divergence between a target distribution defined by the dataset and the marginal distribution of a probabilistic model over the observed variables is minimized.  While the exact gradient of the log-likelihood is often intractable, stochastic approximations can be computed, so long as samples can be drawn from the full probabilistic model, and its posterior given the observed data (cite?).

The efficiency, and thus the applicability, of these techniques is limited by the difficulty of defining practical distributions for which all the requisite sampling operations are computationally efficient.  Sampling is only efficient in undirected graphical models if no loops are present among the connections, severely limiting the range of representable relationships.  In Boltzmann machines, including resticted Boltzmann machines, samples are usually generated using costly but inexact MCMC techniques \cite{hinton1983optimal}.  

In directed models, sampling is only efficient if it can be performed via an ancestral pass.  In particular, it is inefficient to compute the posterior distributions over the hidden causes of observed data if the model consists of a directed acyclic graph in which hidden causes control observed consequences.  %condition on child variables if the parents are not also specified.  
Such models are natural and powerful \cite{hinton2006fast}, but unfortunately, samples from this posterior are required to compute the gradient of the log-likelihood.

\subsection{The expectation lower bound and variational expectation maximization}

In contrast to the exact log-likelihood, it can be comptuationally efficient to optimize a lower bound on the log-likelihood \cite{jordan1999introduction}.  A conventional and mathematically convenient choice is the expectation lower bound (ELBO; $\mathcal{L}(x, \theta, \phi)$), which differs from the true log-likelihood by the KL-divergence between an approximating posterior distribution, $q(z | x, \phi)$, and true posterior distribution, $p(z | x, \theta)$, over the hidden random variables $z$, conditioned on the observed random variables $x$:  
\begin{align}
\mathcal{L}(x, \theta, \phi) &= \log p(x | \theta) - \KL[q(z | x, \phi)||p(z|x,\theta)] \label{variational-inference-equation} \\
&= \int_z q(z | x, \phi) \log\left[ \frac{p(x, z | \theta)}{q(z | x, \phi)} \right] \nonumber %\\
%\KL[q(z | \phi) || p(z|x, \theta)] &= -\int_z q(z|\phi) \log \left[ \frac{p(z|x, \theta)}{q(z | \phi)} \right] \geq 0 \nonumber 
\end{align}
%$\KL(q||p) \geq 0$, so if $\KL(q||p)$ starts small (e.g., $0$), we can safely optimize $\mathcal{L}(q, \theta)$.
The approximating posterior $q(z | x, \phi)$ is designed to be computationally tractable, even though the true posterior $p(z | x, \theta)$ is not.  

The ELBO has a form analogous to $-\KL\left[q(z|x, \phi) || p(x,z|\theta)\right]$, but $p(x,z|\theta)$ is not a distribution over $z$ alone, so this is not a proper KL-divergence.  
Successive optimization of the expectation lower bound with respect to $\phi$ and $\theta$ is called variational expectation-maximization~(EM).
Like Neural Variational Inference and Learning (NVIL) \cite{mnih2014neural}, but unlike traditional variational EM, we define $q(z | x, \phi)$ to be a single, explicit function of $x$, with parameters $\phi$ shared between all inputs $x$.
%We can use variational EM to accomodate the case where the $U$ matrix is not orthogonal by choosing an approximating distribution that matches the energy function of the D-Wave~3.

It is easy to construct a stochastic approximation to variational EM on the ELBO that only requires exact, computationally tractable samples.  Unfortunately, this naive estimate is impractically high-variance, leading to slow training and poor performance \cite{paisley2012variational}.  While the variance of the gradient can be reduced using the baseline technique, originally developed in the reinforcement learning literature, this is dependent upon the choice of an appropriate baseline; a difficult task itself \cite{williams1992simple}.  Prior efforts have trained a neural network to learn a baseline \cite{mnih2014neural}.  

When the distributions are all continuous and decompose in a particular way, the ELBO can be reformulated as a variational auto-encoder using the reparameterization trick \cite{kingma2013auto}.  However, this approach cannot be applied to discrete distributions, since small changes to $q(z|x)$ result in small changes to the weights given to the fixed, exponentially large set of discrete configurations over $z$, rather than small changes to the configurations themselves.  The gradients of $p(x | z)$ cannot be backpropagated into the large, discrete chagnes corresponding to the available configurations.

We investigate this stochastic approximation in general, and show how more sophisticated variance-minimizing baselines, developed for reinforcement learning, can be used in this context.  We also develop a novel variance reduction technique that uses an implicit baseline to completely eliminate the variance due to the imperfection of the approximating posterior.  We show that these implicit baselines produce a lower variance estimate of the gradient than either estimates of the variance-minimizing baseline, or approximations of the baseline using neural networks.  

\section{Stochastic approximations to the gradient of the ELBO}

The gradient of the ELBO with respect to the parameters of the true distribution, $p(x,z|\theta)$, are analytically simple, and the naive stochastic approximation is low-variance:
\begin{equation*}
\frac{\partial \mathcal{L}(\theta, \phi)}{\partial \theta} = \mathbb{E}_{q(z|x)} \left[ \frac{\partial \log p(x,z | \theta)}{\partial \theta} \right]
\end{equation*}
%\begin{align*}
%\frac{\partial \mathcal{L}(\theta, \phi)}{\partial \theta} &= -\int_z q(z|\phi) \cdot \frac{\partial \log p(x,z | \theta)}{\partial \theta} \\
%&= -\int_z \frac{ q(z|\phi) }{p(x,z | \theta)} \cdot \frac{\partial p(x,z | \theta)}{\partial \theta} \\
%&= \mathbb{E}_{q(z|x)} \left[ \frac{1}{p(x,z | \theta)} \cdot \frac{\partial p(x,z | \theta)}{\partial \theta} \right]
%\end{align*}


%Instead of equation~\ref{variational-inference-equation}, 
In contrast, the gradients with respect to the parameters of the approximating posterior, $q(z | x, \phi)$, are more challenging to compute with low variance:
\begin{align}
\frac{\partial \KL (q || p)}{\partial \phi} &= \int_z \frac{\partial q(z | x, \phi)}{\partial \phi}  \cdot \left( \log p(x,z | \theta) - \log q(z | x, \phi) \right) 
- \frac{\partial q(z | x, \phi)}{\partial \phi} \nonumber \\
&= \mathbb{E}_{q} \left[ \left( \log p(x,z | \theta) - \log q(z | x, \phi) - b \right) \cdot \frac{ \partial \log q(z | x, \phi)}{\partial \phi} \right] \label{KL-grad}
\end{align}
where we can both remove $\int_z \frac{\partial q(z | x, \phi)}{\partial \phi}$ and add $b$ because
\begin{equation} \label{mean-zero-grad}
\int_z \frac{\partial q(z | x, \phi)}{\partial \phi} = \frac{\partial}{\partial \phi} \int_z q(z | x, \phi) = \frac{\partial}{\partial \phi} 1 = 0.
\end{equation}
Equation~\ref{KL-grad} is an instance of the REINFORCE algorithm, which was first introduced in the context of reinforcement learning \cite{williams1992simple, bengio2013estimating}.  In the following we thus use the notation
\begin{align*}
R(x,z) &= \log p(x,z | \theta) - \log q(z | x, \phi) - b & \mu_R &= \mathbb{E}_z \left[ \log p(x,z | \theta) - \log q(z | x, \phi) \right] - b \\
r(x,z) &= R(x,z) - \mu_R
\end{align*}
where $R$ can be understood as the reward associated with action $z$ in response to input $x$.  Recently, REINFORCE has been applied to recurrent models of visual attention \cite{mnih2014recurrent, xu2015show}.  Variance reduction has proven to be critical to the success of these applications.  

While $\frac{\partial \log q(z | x, \phi)}{\partial \phi}$ is mean-zero, as is evident from equation~\ref{mean-zero-grad}, $\log p(x,z|\theta) - \log q(z | x, \phi) -b $ is generally not.  As a result, $\mu_R \not= 0$, and the naive stochastic approximation to equation~\ref{KL-grad} is:
\begin{equation} \label{baseline-stochastic-approximation}
\frac{\KL (q || p)}{\partial \phi} \approx \frac{1}{|\mathcal{D}|} \sum_{i \in \mathcal{D}} \left( r_i + \mu_R \right) \cdot \frac{\partial \log q(z|\phi)}{\partial \phi}
\end{equation}
The term $\frac{\mu_R}{|\mathcal{D}|} \cdot \sum_{i \in \mathcal{D}} \frac{\partial \log q(z|\phi)}{\partial \phi}$ has mean zero, but variance
\begin{equation*}
\sigma^2 = \frac{\mu_R^2}{|\mathcal{D} |} \cdot \sigma_{\frac{\partial \log q}{\partial \phi}}^2
\end{equation*}

It is clear that the baseline $b$ must be chose carefully, since $\mathbb{E}_z \left[ \log p(x,z | \theta) - \log q(z | x, \phi) \right]$ may be very large.  Moreover, this term can change significantly as a function of $x$, so $b$ must also be a function of $x$.  This approach is adopted by NVIL, which defines $b(x)$ to be a neural network, and trains it to minimize:
\begin{equation*}
\mathbb{E}_z \left[ \left(R(x,z) \right)^2 \right] = \mathbb{E}_z \left[ \left( \log p(x,z | \theta) - \log q(z | x, \phi) + b(x) \right)^2 \right].
\end{equation*}

If the baseline $b(x)$ is chosen to be the sample mean of $\log p(x,z|\theta) - \log q(z | x, \phi)$, using an independent dataset of size $|\mathcal{D}|$, then $\mu_R$ is mean-zero random variable with variance $\frac{\sigma_R^2}{|\mathcal{D}|}$, and 
\begin{equation*}
\sigma^2 = \frac{1}{|\mathcal{D} |} \cdot \sigma_R^2 \cdot \sigma_{\frac{\partial \log q}{\partial \phi}}^2
\end{equation*}
(WORK OUT IN DETAIL).  This baseline was proposed by Sutton (CHECK) \cite{sutton1984temporal}, and analyzed by Weaver \& Tao \cite{weaver2001optimal} in an input-independent reinforcement learning context, where $b$ is not a function of $x$.  It is optimal in temporal systems when reward is not discounted, and the average reward must be optimized using a single, constant baseline.  %(Weaver \& Tao yields the same result as Greensmith when rewards are immediate.)  
However, this analysis only holds so long as independent samples are used to estimate the sample mean of $\log p(x,z|\theta) - \log q(z | x, \phi)$.

Surprisingly, the input-dependent baseline $b(x)$ that minimizes the variance of gradient is not the expected reward \cite{weaver2001optimal, greensmith2004variance}.  Rather we can derive:
\begin{align*}
\sigma_{b(x)}^2 &= \mathbb{E}_q \left[ \left( \left( \log p - \log q - b(x) \right) \cdot \frac{\partial \log q}{\partial \phi}  %\right. \right. \\
%&\qquad \left. \left. 
- \mathbb{E}_q \left[ \left( \log p - \log q - b(x) \right) \cdot \frac{\partial \log q}{\partial \phi} \right] \right)^2
\right] \\
&= \mathbb{E}_q \left[ \left( \left( \log p - \log q \right) \cdot \frac{\partial \log q}{\partial \phi}  
- \mathbb{E}_q \left[ \left( \log p - \log q \right) \cdot \frac{\partial \log q}{\partial \phi}  \right] \right. \right. \\
&\qquad \left. \left. - b(x) \cdot \frac{\partial \log q(z| \phi)}{\partial \phi} + \mathbb{E}_q \left[ b(x) \cdot \frac{\partial \log q}{\partial \phi} \right] \right)^2 \right] \\
&= \sigma_{b = 0}^2 + \mathbb{E}_q \left[ \left( b(x) \cdot \frac{\partial \log q(x|\phi) }{ \partial \phi } \right)^2
- 2 \cdot \left(b \cdot \frac{ \partial \log q}{ \partial \phi } \right) \cdot \left( \left( \log p - \log q \right) \cdot \frac{\partial \log q}{\partial \phi} \right) \right] \\
&= \sigma_{b = 0}^2 + b(x)^2 \cdot \mathbb{E}_q \left[ \left( \frac{\partial \log q(x|\phi) }{ \partial \phi } \right)^2 \right]
- 2 \cdot b(x) \cdot  \mathbb{E}_q \left[ \left( \log p - \log q \right) \cdot \left( \frac{\partial \log q}{\partial \phi} \right)^2 \right]
\end{align*}
where we have suppressed the arguments of $p$ and $q$ to improve clarity, and the third line follows from equation~\ref{mean-zero-grad}.  Taking the derivative of $\sigma_{b(x)}^2$ with respect to $b(x)$ and setting it equal to zero, it is apparent that $\sigma_{b(x)}^2$ is minimized when:
\begin{equation} \label{minimal-variance-baseline}
b_{varmin}(x) = \frac{\mathbb{E}_q \left[ \left( \frac{ \partial \log q(z|\phi)}{\partial \phi} \right)^2 \cdot \left( \log p(x,z | \theta) - \log q(z | x, \phi) \right) \right]}
{ \mathbb{E}_q \left[ \left( \frac{ \partial \log q(z | x, \phi) }{\partial \phi} \right)^2 \right] } .
\end{equation}
It is worth noting, though, that the variance of a stochastic approximation to $b_{varmin}(x)$ will itself be large.  As a result, any stochastic approximation to the gradient based upon such an estimated baseline will be high (EXPAND).

\section{Matched-sampling} \label{matched-sampling-section}

Any deterministic baseline $b(x)$ will yield an unbiased estimator for equation~\ref{KL-grad}.  However, $b(x)$ must be independent of the samples used in the stochastic approximation of the expectation; otherwise, $b(x)$ cannot be factored out of the expectation, allowing the application of equation~\ref{mean-zero-grad}.   

In an effort to eliminate the variance in the estimator of equation~\ref{baseline-stochastic-approximation} caused by $\mu_R$, without explicitly estimating a baseline, we introduce the additional 
\begin{equation*}
-\mathbb{E}_{q} \left[ \left( \log p - \log q - b \right) \right] \cdot \mathbb{E}_{q} \left[ \frac{ \partial \log q}{\partial \phi} \right]
\end{equation*}
into equation~\ref{KL-grad}.  This term follows naturally from the analysis of the gradient for undirected graphical models in appendix~\ref{undirected-models-section}.  It leaves the gradient estimate unbiased, since it is mean-zero: taking advantage of equation~\ref{mean-zero-grad},
\begin{equation*}
\mathbb{E}_{q} \left[ \frac{ \partial \log q}{\partial \phi} \right]  = \int_z \frac{\partial q}{\partial \phi} = 0.
\end{equation*}

%we can compute an implicit baseline using the same samples as the rest of the stochastic approximation, and in the process exactly eliminate $\mu_R$.
Combining this term with equation~\ref{KL-grad}, we obtain.
\begin{equation}
\frac{\partial \KL (q || p)}{\partial \phi} = \mathbb{E}_{q} \left[ \left( \log p - \log q - b \right) \cdot \frac{ \partial \log q}{\partial \phi} \right] 
- \mathbb{E}_{q} \left[ \left( \log p - \log q - b \right) \right] \cdot \mathbb{E}_{q} \left[ \frac{ \partial \log q}{\partial \phi} \right].
\label{matched-sampling}
\end{equation}
The stochastic approximation is now:
\begin{align}
\frac{\partial \KL(q||p)}{\partial \phi}  &\approx 
\frac{1}{|\mathcal{D}|} \sum_{i \in \mathcal{D}} \left( \left(r_i + \mu_R\right) \cdot \frac{\partial \log q_i}{\partial \phi} \right) - \left( \frac{1}{|\mathcal{D}|} \sum_{i \in \mathcal{D}} (r_i + \mu_R) \right) \cdot \left( \frac{1}{|\mathcal{D}|} \sum_{i \in \mathcal{D}} \frac{\partial \log q_i}{\partial \phi} \right) \nonumber \\
&\approx \frac{1}{|\mathcal{D}|} \cdot \left( \sum_{i \in \mathcal{D}}\left(r_i \cdot \frac{\partial \log q_i}{\partial \phi} \right) +  \mu_R \sum_{i \in \mathcal{D}} \frac{\partial \log q_i}{\partial \phi} \right) \nonumber \\
&\qquad - \left( \frac{1}{|\mathcal{D}|} \sum_{i \in \mathcal{D}} r_i \right) \cdot \left( \frac{1}{|\mathcal{D}|} \sum_{i \in \mathcal{D}} \frac{\partial \log q_i}{\partial \phi} \right) + \frac{\mu_R}{|\mathcal{D}|} \sum_{i \in \mathcal{D}} \frac{\partial \log q_i}{\partial \phi}  \nonumber \\
&\approx \frac{1}{|\mathcal{D}|} \sum_{i \in \mathcal{D}}\left( r_i \cdot \frac{\partial \log q_i}{\partial \phi} \right) - \left( \frac{1}{|\mathcal{D}|} \sum_{i \in \mathcal{D}} r_i \right) \cdot \left( \frac{1}{|\mathcal{D}|} \sum_{i \in \mathcal{D}} \frac{\partial \log q_i}{\partial \phi} \right) \label{KL-grad-matched-sampling}
\end{align}
Notably, $\frac{\mu_R}{|\mathcal{D}|} \sum_{i \in \mathcal{D}} \frac{\partial \log q_i}{\partial \phi}$ cancels exactly between the two halves, even if the dataset $\mathcal{D}$ is small.  As a result, there is not need to include an explicit baseline.  We refer to this approach as the \emph{matched-sampling} estimate.

The matched-sampling estimate is analogous to using the sample mean of the ``reward,''
\begin{equation*}
\frac{1}{|\mathcal{D}|} \sum_{i \in \mathcal{D}} \log p(x,z_i | \theta) - \log q(z_i | x, \phi),
\end{equation*}
as the baseline.  
This can be seen by rewriting equation~\ref{KL-grad-matched-sampling} as:
\begin{equation*}
\frac{\partial \KL(q||p)}{\partial \phi}  \approx 
\frac{1}{|\mathcal{D}|} \sum_{i \in \mathcal{D}}\left( \left[ \log p_i - \log q_i - \left( \frac{1}{|\mathcal{D}|} \sum_{i \in \mathcal{D}} \log p_i - \log q_i \right) \right]  \cdot \frac{\partial \log q_i}{\partial \phi} \right)
\end{equation*}
Surprisingly, this sample mean is evaluated over the same samples as the stochastic approximation of the gradient.  
The standard derivation of the baseline does not support the use of this sample mean, since it is correlated with $\frac{\partial \log q(z | x, \phi)}{\partial \phi}$ in equation~\ref{KL-grad}, and so cannot be factored out of the expectation.  Nevertheless, equation~\ref{KL-grad-matched-sampling} shows this baseline to be acceptable.

We can cancel matching terms between the halves of equation~\ref{KL-grad-matched-sampling} to obtain:
\begin{equation} \label{KL-grad-matched-sampling-cancel}
\frac{\partial \KL(q||p)}{\partial \phi}  \approx \frac{|\mathcal{D}| - 1}{|\mathcal{D}|^2} \sum_{i \in \mathcal{D}}\left( r_i \cdot \frac{\partial \log q_i}{\partial \phi} \right) - \left( \frac{1}{|\mathcal{D}|} \sum_{i \in \mathcal{D}} r_i \right) \cdot \left( \frac{1}{|\mathcal{D}|} \sum_{j \in \mathcal{D} \setminus i} \frac{\partial \log q_i}{\partial \phi} \right)
\end{equation}
The second term of equation~\ref{KL-grad-matched-sampling-cancel} is a product of independent, mean-zero random variables, and so has variance
\begin{equation*}
\frac{|\mathcal{D} - 1|}{|\mathcal{D}|^3} \cdot \sigma_R^2 \cdot \sigma_{\frac{\partial \log q}{\partial \phi}}^2
\end{equation*}

Unlike NVIL or the variance-minimizing baseline, the matched-sampling estimate reduces to zero if $|\mathcal{D}| = 1$.  As a result, when evaluating matched-sampling, we always use at least two samples from the posterior distribution.  We show in section~\ref{experimental-section} that the reduction in variance of the matched-sampling estimate more than compensates for the additional required computation, leading to faster training.  

\section{Experimental results} \label{experimental-section}

We are primarily interested in variance reduction of the gradient $\frac{\partial \KL (q || p)}{\partial \phi}$ of equation~\ref{KL-grad}.  A low-variance gradient is necessary to learn a good probabilistic model \cite{paisley2012variational, mnih2014neural}.  The quality of the final learned model is also affected by a number of largely independent choices, including the probabilistic model itself and the optimization algorithm that uses the gradients.  For this reason, we examine our variance-reduction techniques in the context of an extremely simple system: sigmoid belief networks with one hidden layer \cite{neal1992connectionist}.  Moreover, we focus our analysis on the variance of the gradient throughout the training trajectory, rather than the log-likelihood of the final learned network.

We evaluate the variance of the gradients produced by NVIL \cite{mnih2014neural}, the stochastic approximation to the variance-minimizing baseline \cite{weaver2001optimal, greensmith2004variance}, and the implicit matched-sampling baseline.  The variance-minimizing baseline and matched-sampling produce much lower-variance estimates of the gradient than NVIL on a simple example model.  Matched-sampling achieves superior performance than the variance-minimizing baseline, despite requiring half as many samples.


\subsection{Model system}
   

We trained sigmoid belief networks of 200 binary random variables on the binarized MNIST dataset~\cite{salakhutdinov2008quantitative}.  The models were trained using NVIL \cite{mnih2014neural}, both following the optimization procedure described in that paper, and using ADADELTA \cite{zeiler2012adadelta} without regularization until validation performance stopped improving.  The final lower bound on the negative log-likelihood of the test set is approximately $-122$.
Our probablistic model and approximating posterior are defined by:
\begin{align*}
p(z_i = 1) &= \theta_i^z &
p(z) &= \prod_i p(z_i) \\
p(x_i=1 |z, \theta) &= \sigma \left( \theta_i^x + \sum_j \theta_{i,j} \cdot z_j \right) &
p(x) &= \prod_i p(x_i) \\
q(z_i = 1 | x) &= \sigma \left( \phi_i^z + \sum_j \phi_{i,j} \cdot x_j \right) &
q(z | x) &= \prod_i q(z_i | x)
\end{align*}
where $\sigma(x) = (1 + e^{-x})^{-1}$.


\begin{figure}[tb]
  \begin{center}
    \begin{tabular}{p{0.0in}p{2.25in}p{0.0in}p{2.25in}}
      \parbox[b]{0in}{(a) \vspace{3.5cm}} & \includegraphics[width=2.5in, keepaspectratio=true]{normalized_MSE} & 
      \parbox[b]{0in}{(b) \vspace{3.5cm}} & \includegraphics[width=2.5in, keepaspectratio=true]{time_all_samples}  %5.5in
    \end{tabular}
  \end{center}
  \caption{(a) Ratio of the standard deviation to the magnitude of the gradient using matched-sampling, as a function of training epoch (x axis) and number of samples from the posterior (color).  (b) Time required to compute the gradient, as a function of number of samples from the posterior (x axis) and gradient estimation method (color).   \label{ms_learning_and_cpu_time}} 
\end{figure}

\begin{figure}[tb]
  \begin{center}
    \begin{tabular}{p{0.0in}p{2.25in}p{0.0in}p{2.25in}}
      \parbox[b]{0in}{(a) \vspace{3.5cm}} & \includegraphics[width=2.5in, keepaspectratio=true]{MSE_normalized_2samples} & 
      \parbox[b]{0in}{(b) \vspace{3.5cm}} & \includegraphics[width=2.5in, keepaspectratio=true]{MSE_normalized_12samples}  %5.5in
    \end{tabular}
  \end{center}
  \caption{Ratio of the standard deviation to the magnitude of the gradient, as a function of training epoch (x axis) and gradient estimation method (color), using 2 samples (a) or 12 samples (b) from the posterior.  \label{learning_fixed_samples}} 
\end{figure}


The variance of the gradient is due to a subset of the following three sources: the variance of the baseline around its optimal value, the bias of the baseline from its optimal value, and the variance of the stochastic estimator given the optimal baseline \cite{greensmith2004variance}.  NVIL uses a zero-variance (deterministic) but biased estimate of the optimal baseline; the variance-minimizing baseline method uses a zero-bias stochastic estimate of the optimal baseline; matched-sampling uses an implicit stochastic baseline.  To characterize the collective effect of these sources of variance, %and account for the possiblity that the bias of NVIL's baseline 
we measure the variance of the gradient for randomly selected elements of the dataset throughout the course of training.  

%Measuring the variance with respect to each element of the dataset separately would also significantly reduce the measured variance, since it separates out the variance of the gradient for a single element of the dataset due to the stochastic estimation procedure, from the variance of these gradients over the dataset.  Are these two sources of variance not uncorrelated?

To evaluate the variance of the gradient, we first compute an estimate of the ground-truth gradient $\hat{g} = \frac{\partial \mathcal{L}}{\partial \phi}$ %restricted to $100$ elements of the training dataset $\{x_1, x_2, ..., x_{100}\}$
for the chosen elements of the dataset. The gradient for each element of the dataset $x$ is estimated using $10^4$ samples from $q(z|x, \phi)$, with %either the stochastic approximation of equation~\ref{baseline-stochastic-approximation} and the variance-minimizing baseline of equation~\ref{minimal-variance-baseline}, or 
the unbiased matched-sampling implicit baseline technique of equation~\ref{KL-grad-matched-sampling}.  The standard deviation of these estimates of the true gradient are no more than ??? percent of their magnitude (KYUNGHYUN!!!).  

Given this ground truth, we then evaluate the variance of each estimator on 20 (???) elements of the datset using 1, 2, 4, or 8 samples from $q(z|x, \phi)$.\footnote{We do not use 1 sample with the matched-sampling technique, since in this case it degenerates to a constant value of $0$.}  
When using the variance-minimizing baseline, the baseline itself is calculated with an independent set of samples of the same size, so the total number of samples required is twice that used for the final gradient estimation.
For each element of the dataset, we compute the approximate gradient 20 (???) times using independent samples from $q(z|x, \phi)$.  For each element of the dataset, the variance is estimated using
\begin{equation*}
\text{var}_k \approx \frac{1}{20} \sum_{1 \leq j \leq 20} \left\|g_k^j - \hat{g_k} \right\|^2, 
\end{equation*}
and the average variance is approximated by $\text{var} \approx \frac{1}{20} \sum_{1 \leq k \leq 20} \text{var}_k$,
%\begin{equation*}
%\text{var} = \frac{1}{20 \cdot 20} \sum_{1 \leq j,k \leq 20} \left\|g_k^j - \hat{g_k} \right\|^2, 
%\end{equation*}
where $g_k^j$ is the $j$th estimated gradient for element $k$ of the dataset.  %Since the variance-minimizing baseline and matched-sampling implicit baseline technique are unbiased, $MSE$ is the variance of the estimator.  %NVIL, in contrast, is biased, 
To evaluate the relative effect of the variance on gradient-based update, we report $\frac{\sqrt{\text{var}_k}}{|\hat{g}|}$, which captures the noise-to-signal ratio.

In practice, the additional samples required to use the variance-minimizing or matched-sampling baselines do not take much time to compute.  In a single-layer sigmoid belief network, the distribution $q(z|x, \phi)$ is a product of Bernoulli distributions.  To generate multiple samples from $q(z|x, \phi)$, the distribution need only be calculated once; each sample only requires the generation of $n$ random numbers, where $n$ is the number of hidden variables.  


\begin{comment}
(3) MSE for each baseline technique
  (a) Compute the gradient estimate $g_k^j$ using randomly select 10 inputs out of the 100 inputs $\{x_1, x_2, .., x_{10}\}$ and 10 samples from $q(z|x)$ for each $x_t$
  (b) Repeat (a) for 20 times with the fixed input set to get $\{g_k^1, .., g_k^{20}\}$
  (c) Repeat (a-b) for 20 times to get $\{ g_1^1, g_1^2, .., g_{20}^{20}\}$
  (d) Compute the squared error $\text{SE}_k^j = \| g_k^j - \hat{g} \|_2^2$ for all $k=1,..,20$ and $j=1,..,20$
  (e) The final MSE is $\text{MSE} = \frac{1}{20*20} \sum_{k,j=1}^{20} \text{SE}_k^j,$ and below I also report the standard deviation of the squared errors.
\end{comment}


%The MSE corresponds to the variance: $\mathcal{E}[(g - \mathcal{E}[g])^2]$. I will compute the bias $((\mathcal{E}[g] - g)^2)$ for each estimator and get back to you shortly. VAR is nothing but $Var[(g - E[g])^2]$ to see the variance of the empirical estimate of the variance of the gradient estimate. :)

\subsection{Results}

Throughout the training process, the standard deviation of the gradient decreases relative to its magnitude as more samples are taken from the posterior when estimating the gradient, as shown in figure~\ref{ms_learning_and_cpu_time}(a).  The time required to compute and use these samples is approximately linear in the number of samples, and varied little between gradient approximation techniques, as demonstrated by figure~\ref{ms_learning_and_cpu_time}(b).

Figure~\ref{learning_fixed_samples} shows that, regardless of the number of samples taken from the posterior, matched-sampling is strictly superior to the variance-minimizing baseline, and the NVIL gradient estimate has considerably higher variance than the other two methods.  





\begin{comment}
\section{Inapplicability of auto-encoding variational Bayes}

When the distributions are all continuous and decompose in a particular way, the variational auto-encoder\footnote{Kingma} (VAE) efficiently minimizes the ELBO of equation~\ref{variational-inference-equation} by using the reparameterization trick.
The VAE regroups the variational loss function as:
\begin{align}
\mathcal{L}_{VAE}(x, \theta, \phi) &= \log p(x | \theta) - \KL[q(z | x, \phi)||p(z|x,\theta)] \nonumber \\
&= \int_z q(z|\phi) \cdot \left(\log p(z|\theta) - \log q(z|\phi) \right) + \int_z q(z|\phi) \cdot \log p(x|z, \theta) \nonumber \\
&= -\KL\left[q(z|\phi) || p(z|\theta) \right] + \mathbb{E}_q \left[ \log p(x| z, \theta) \right]. \label{vae-equation}
\end{align}
The reparameterization trick can be applied if $q(z|\phi)$ can be formulated as a differentiable, deterministic function of the combination of the inputs, the parameters, and a set of input- and parameter-independent random variables.
For instance, given a Gaussian distribution with mean and variance determined by the input, 
\begin{equation*}
\mathcal{N}\left(m(x), v(x) \right) \sim m(x) + \sqrt{v(x)} \cdot \mathcal{N}\left(0, 1 \right) .
\end{equation*}
The reparameterization trick changes the distribution over which the expectation is taken in equation~\ref{vae-equation} to that of the independent random variable.  The argument of the expectation is thus rendered a single, continuous auto-encoding function conditioned on this random variable.  As a result, backpropagation can be applied, conditioned on the random variable.

In contrast, when the random variables are discrete rather than continouous, the reparameterization trick cannot be applied directly without significant approximations.\footnote{But cite Bengio}  Small changes to the encoder $q(z|\phi)$ result in small changes to the weights given to the fixed, exponentially large set of discrete configurations, rather than small changes in the configurations themselves.  The gradients of the decoder cannot be backpropagated into the large, discrete changes corresponding to the available configurations.  
\end{comment}


\begin{comment}
Our approach differs from a variational autoencoder in the grouping of the loss function.  Variational autoencoders efficiently backpropagate gradients of the loss function when the approximating posterior $q(z|x)$ consists of a differentiable, deterministic function of the combination of the inputs, the parameters, and a set of input- and parameter-independent random variables.  For instance, given a Gaussian distribution with mean and variance determined by the input, 
\begin{equation*}
\mathcal{N}\left(m(x), v(x) \right) \sim m(x) + \sqrt{v(x)} \cdot \mathcal{N}\left(0, 1 \right) .
\end{equation*}
In contrast, a comparable formulation is not possible for the distribution of equation~\ref{D-Wave-distribution}, implemented by the D-Wave~3.  
\end{comment}




\section{Discussion}

We have presented a novel technique to compute the gradient of the expectation lower bound (ELBO) for unsupervised training of probabilistic models, and have shown that it exhibits lower variance than the state-of-the-art methods.  In appendix~\ref{undirected-models-section}, we show that this technique can be trivially extended to undirected graphical models, for which the partition function is intractable.  

\pagebreak

\bibliographystyle{unsrt} % abbrvnat

\bibliography{matched_sampling,ml}

\appendix

\pagebreak

\section{Differentiating the ELBO with undirected graphical models} \label{undirected-models-section}

%While the extremely low-variance variational auto-encoder cannot be applied with discrete random variables, 

%The ELBO can be differentiated directly, but the naive stochastic approximation to this calculation yields a high-variance estimator.



%We develop an alternative formulation of the gradient of the ELBO, which produces efficient gradients for discrete random variables.  Gradients are not backpropagated directly from the decoder into the encoder, since changes to the encoder yield small changes in the probability of very different states, rather than small changes to the states themselves.  However, the gradients estimated from the covariance of equation~\ref{KL-gradient-cov} are efficiently propagated into the deep structure of the encoder.  We extend to a mixture of continuous and discrete random variables in section~\ref{quantum-VAE-section}.

The matched sampling approach of section~\ref{matched-sampling-section} can easily be generalized to undirected graphical models.
We identify $q$ with a distribution over $z$, conditioned on $x$, that can be sampled efficiently; and $p$ with a distribution over $z$ and $x$ that can be sampled efficiently.  %qDBN with non-orthogonal $U$ matrix.  %In section~\ref{qDBN-implementation}, we will choose $q$ to be the ``approximate'' posterior of $p$ that results from assuming that the $U$ matrix is orthogonal, even if it is not.  
The approximate posterior distribution $q$ and the exact posterior distribution of $p$ can be represented with the form:
\begin{equation} \label{energy-model-equation}
q(z|x, \phi) = \frac{e^{-E_q(z|x, \phi)}}{\int_z e^{-E_q(z|x, \phi)}} \hspace{1.5cm}
p(z|x, \theta) = \frac{e^{-E_p(z|x, \theta)}}{\int_{z} e^{-E_p(z|x, \theta)}}  
\end{equation}
In the interest of clarity, we often drop the dependence on $x$ and the parameters.  
%Since $q$ is a function of the input, and the true (variational) loss function is the sum of $\mathcal{L}$ over all elements of the dataset, we use a single set of parameters $\phi$ for all elements of the dataset, and choose a parametric form that depends upon the input, corresponding to the intractable $p(z|x)$. 
%Indeed, $q$ may share parameters with $p$, allowing the approximate posterior distribution to change immediately as $p$ is trained.  To denote this implicit dependence on the input, we sometimes write $q(z|x, \phi)$ below.  

%THIS DOESN'T WORK!  Taking the log of $p(x,z|\theta)$ is not simple, since we can't sum the exponent.  In taking the derivative with respect to $\theta$, we need to explicitly differentiate the log of the product of the exponentiated terms,   

%To perform variational EM, $\mathcal{L}(\theta, \phi)$ is minimized with respect to $\phi$ and $\theta$ via coordinate gradient descent; that is, $\mathcal{L}$ is subject to gradient descent first with respect to $\phi$ and then with respect to~$\theta$.  Since each E and M step increases the expectation lower bound, the resulting algorithm will converge to at least a local maximum, as per Neal \& Hinton (1998).\footnote{A view of the EM algorithm that justifies incremental, sparse, and other variants}  If we initialize $\KL[q(z | x, \phi) || p(z | x, \theta)]$ to be zero, such as by choosing the starting $U$ to be orthogonal in section~\ref{qDBN-implementation}, then the final increase of the log-likelihood is lower-bounded by the growth of the ELBO.  

The derivative of $\mathcal{L}(x, \theta, \phi)$ with respect to $\theta$ (i.e., with respect to $p$) corresponds to using $q(z|x,\phi)$ to sample the hidden units $z$ in the positive phase, and using $p$ to sample the hidden units in the negative phase; the log is over the full-state $p$, rather than being marginalized over the hidden units; the partition function remains unchanged and independent of $q$.  
\begin{align}
\frac{\partial \mathcal{L}(\theta, \phi)}{\partial \theta} &= -\int_z q(z|\phi) \cdot \frac{\partial E_p(x, z| \theta)}{\partial \theta} + \int_{x, z} p(x,z|\theta) \cdot \frac{\partial E_p(x, z | \theta)}{\partial \theta} \nonumber \\
&= -\mathbb{E}_{q(z|x)} \left[ \frac{\partial E_p(x,z)}{\partial \theta} \right] + \mathbb{E}_{p(x,z)} \left[ \frac{\partial E_p(x,z)}{\partial \theta} \right] \label{KL-gradient-p}
\end{align}


The derivative of $\mathcal{L}(\theta, \phi)$ with respect to $\phi$ only requires samples with respect to $q(z|x,\phi)$.  
\begin{equation*}
\frac{\partial}{\partial \phi} \mathcal{L} = -\frac{\partial}{\partial \phi} \KL[q(z|\phi)||p(z|x,\theta)]
\end{equation*}
so we are free to use $E_p(z|x)$.  While it is intractable to sample from this posterior distribution, it is easy to calculate its energy.
%For this calculation, it is essential that we use $p(x,z | \theta)$, which is tractable; rather than $p(z|x,\theta)$, which is not.  %Fortunately, $p(Z|X,\theta)=\frac{e^{-E_p(x,z)}}{\int_z e^{-E_p(x,z)}}$, so the energy corresponding to the conditional probability is identical to that of the joint probability; only the partition function differs, which we shall see does not affect the gradient.
%As a result, we should more properly write $p(z;x) = \frac{e^{-E_p(x,z)}}{\int_{x,z} e^{-E_p(x,z)}}$.  Keeping this in mind,
\begin{align}
\KL(q || p) &= \int_z \frac{e^{-E_q(z)}}{\int_z e^{-E_q(z)}} \cdot \log \left( \frac{e^{-E_q(z)}}{\int_z e^{-E_q(z)}} \right) - 
\int_z \frac{e^{-E_q(z)}}{\int_z e^{-E_q(z)}} \cdot \log \left( \frac{e^{-E_p(z)}}{\int_z e^{-E_p(z)}} \right) \nonumber \\
%\end{align}
%\begin{align}
\frac{\partial \KL(q || p)}{\partial \phi} &= \int_z \left[ \log \left( \frac{e^{-E_q(z)}}{\int_z e^{-E_q(z)}} \right) 
- \log \left( \frac{e^{-E_p(z)}}{\int_z e^{-E_p(z)}} \right) + 1 \right] \nonumber \\
&\quad\qquad \cdot \left[ -\frac{\partial E_q(z)}{\partial \phi} \cdot \frac{e^{-E_q(z)}}{\int_z e^{-E_q(z)}} 
+ \frac{e^{-E_q(z)} \cdot \int_z \left( \frac{\partial E_q(z)}{\partial \phi} \cdot e^{-E_q(z)} \right)}{\left( \int_z e^{-E_q(z)} \right)^2} 
\right] \nonumber \\
&= \mathbb{E}_q \left[ \left(-E_q(z) + E_p(z) - \log \mathcal{Z}_q + \log \mathcal{Z}_p + 1 \right) \cdot 
\left( -\frac{\partial E_q(z)}{\partial \phi} + \mathbb{E}_q \left[ \frac{ \partial E_q(z)}{\partial \phi} \right] \right) \right] \nonumber \\
&= -\mathbb{E}_q \left[ \left(-E_q(z) + E_p(z) - \log \mathcal{Z}_q + \log \mathcal{Z}_p + 1 \right) \cdot \frac{\partial E_q(z)}{\partial \phi} \right] \nonumber \\
&\qquad + \mathbb{E}_q \left[ -E_q(z) + E_p(z) - \log \mathcal{Z}_q + \log \mathcal{Z}_p + 1 \right] 
\cdot \mathbb{E}_q \left[ \frac{\partial E_q(z)}{\partial \phi} \right] \nonumber \\
&= -\mathbb{E}_q \left[ \left(-E_q(z) + E_p(z) \right) \cdot \frac{\partial E_q(z)}{\partial \phi} \right]
+ \mathbb{E}_q \left[ -E_q(z) + E_p(z) \right] \cdot \mathbb{E}_q \left[ \frac{\partial E_q(z)}{\partial \phi} \right] \label{KL-gradient-two-terms}\\
&= \cov_q \left( E_q(z) - E_p(z), \frac{\partial E_q(z)}{\partial \phi} \right) \label{KL-gradient-cov} \\
&= \mathbb{E}_q \left[ \left( \mathbb{E}_q\left[-E_q(z) + E_p(z)\right] - \left(-E_q(z) + E_p(z) \right)\right) \cdot \frac{\partial E_q(z)}{\partial \phi} \right] \label{KL-gradient-one-term}
\end{align}
%where $\log \mathcal{Z}_q = \log \left( \int_x e^{-E_p(x|z)} \cdot \int_z e^{-E_p(z)} \right)$
where equation~\ref{KL-gradient-two-terms} follows since the expectation with respect to $q$ of a constant, such as $\log \mathcal{Z}_q$, factors out of the expectation and cancels between the two terms.
%Importantly, unlike the case investigated by Paisley, Blei, \& Jordan (2012),\footnote{Variational Bayesian inference with stochastic search} which motivated the framework of Kingma \& Welling (2014),\footnote{Auto-encoding variational Bayes} the gradient of the log probability (implicit in the derivative of an expectation with regards to its distribution) cancels out, and $\frac{\partial KL(q||p)}{\partial \phi}$ has low variance.

\end{document}
